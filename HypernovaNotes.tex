%%%%%%%%%%%%%%%%%%%%%%%%%%%%%%%%%%%%%%%%%%%%%%%%%%%%%%%%%%%%%%%%%%%%%%
\NeedsTeXFormat{LaTeX2e}
\documentclass[a4paper,10pt]{article}
\pdfoutput=1

%-- used packages ------------------------------------------------------

\usepackage[utf8]{inputenc}

\usepackage{amsmath}
\usepackage{amssymb}
%\usepackage{epsfig}
%\usepackage{graphicx}
%\usepackage{cite}
\usepackage{multirow}
\usepackage{longtable}
\usepackage{lscape}
\usepackage{color}
\usepackage{hyperref}
% \usepackage{srcltx}


%-- page parameters -------------------------------------------------

\jot = 1.5ex
\parskip 5pt plus 1pt
\parindent 0pt
\evensidemargin -0.1in   \oddsidemargin  -0.1in
\textwidth  6.45in       \textheight 9.1in
\topmargin -1.0cm        \headsep    1.0cm


%opening
\title{Extended Hypernova Notes}

\author{Nicholas Senno, Philipp Baerwald, and Pet\'{e}r M\'{e}sz\'{a}ros}
%\affiliation{Department of Astronomy and Astrophysics; Department of Physics; Center for Particle and Gravitational Astrophysics; Institute for Gravitation and the Cosmos; \\ Pennsylvania State University, 525 Davey Lab, University Park, PA 16802, USA}

%\author{Philipp Baerwald}
%\affiliation{Department of Astronomy and Astrophysics; Department of Physics; Center for Particle and Gravitational Astrophysics; Institute for Gravitation and the Cosmos; \\ Pennsylvania State University, 525 Davey Lab, University Park, PA 16802, USA}

%\author{Pet\'{e}r M\'{e}sz\'{a}ros}
%\affiliation{Department of Astronomy and Astrophysics; Department of Physics; Center for Particle and Gravitational Astrophysics; Institute for Gravitation and the Cosmos; \\ Pennsylvania State University, 525 Davey Lab, University Park, PA 16802, USA}

\begin{document}

\maketitle

\begin{abstract}
	This is a short summary of what we have been doing so far in the Hypernova project.
\end{abstract}

\section{Basic Idea}

We want to take a look at Hypernovae as possible sources of cosmic rays (CR) and high energy neutrinos. Hypernovae are more luminous versions of supernovae and are also assumed to be connected to GRBs. A basic description of the physics of Hypernovae can be taken from the supernova model given by Spitzer~\cite{1998ppim.book.....S}. Based on this model we are mainly interested in the Sedov-Taylor-phase of adiabatic expansion. After calculating the result for particles escaping the source, we will then take into account the propagation of the CR through the host galaxy, the intergalactic medium, and finally our own galaxy. Hence we will devide the calculation in four parts which are assumed to factorize.
In the next sections, we will discuss the different parts in more detail.

\section{A model for Hypernovae}

As mentioned before, we assume that Hypernovae are basically very energetic supernova, which eject matter with a higher kinetic energy.

\bibliographystyle{plain}
\bibliography{References}

\end{document}
